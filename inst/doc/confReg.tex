
\documentclass{article}

\usepackage{amsmath}

\usepackage{natbib}
\usepackage{indentfirst}
\usepackage{Sweave}


% \VignetteIndexEntry{confReg -- Examples}

\begin{document}

\title{Estimating Confidences of Individual Regression Predictions with confReg}
\author{Sebastian Briesemeister}
\maketitle

\section{Introduction}

In contrast to classification approaches, where the uncertainty 
between classes is an indicator of reliability, most regression
methods do not offer an opportunity to estimate the confidence of
individual predictions.

In this package, we wrap different methods for confidence estimation
for regression prediction. In addition, the package implements a linear
regression, a ridge regression, a support vector regression, and a
simple feature selection approach. Every confidence
estimator measures the confidence in a prediction using a confidence
score between zero and one, where a confidence score of zero 
corresponds to a low-confidence prediction and a score close to one
corresponds to a highly confident prediction. In addition, confidence
scores are translated into confidence intervals, in which the
prediced response lies with a probability of $80\%$. Given a good
confidence estimation, the confidence intervals will be tighter for
confident predictions.

To load the package type:

\begin{Schunk}
\begin{Sinput}
> library("confReg")
\end{Sinput}
\end{Schunk}


\section{An Artificial Dataset}

We first create an artificial dataset for testing the described
methods of this package. We create a dataset of 200 instances
with 10 features and one response. The response is created by applying
the Friedman function to the first five features and adding some noise.

\begin{Schunk}
\begin{Sinput}
> set.seed(100)
> dataMatrix <- matrix(runif(10 * 200), 200, 10)
> y <- (10 * sin(pi * dataMatrix[, 1] * dataMatrix[, 2]) + 20 * 
+     (dataMatrix[, 3] - 0.5)^2 + 10 * dataMatrix[, 4] + 5 * dataMatrix[, 
+     5]) + rnorm(200, 0, 0.1)
> dataMatrix <- cbind(dataMatrix, y)
\end{Sinput}
\end{Schunk}

After creating the data matrix, we save the data in a Dataset object
of the package.

\begin{Schunk}
\begin{Sinput}
> data <- Dataset(dataMatrix)
\end{Sinput}
\end{Schunk}

Since we have to test our methods, we split the dataset in
a training and test dataset. We calculate the fold assignment for
every instance assuming a split in five parts and assign set five
as our test set.

\begin{Schunk}
\begin{Sinput}
> folds <- calculateFolds(data, 5, random = TRUE)
> datasets <- splitByFolds(data, folds, 5)
\end{Sinput}
\end{Schunk}

Finally, we normalize the training dataset to zero mean and a
standard deviation of one and apply the same normalization to
the test dataset.

\begin{Schunk}
\begin{Sinput}
> trainingData <- normalize(datasets[[1]])
> testData <- normalizeBy(datasets[[2]], trainingData)
\end{Sinput}
\end{Schunk}


\section{Model Creation}

Lets create a model on our training data. For this, we
will first select features since some features in the dataset
are non-predictive. We first create a regression object and
a model evaluation object, which is needed for evaluating
the worth of a feature set in the feature selection.
The regression object can be created with any regression
method that uses a formula as input for training, e.g.\ ``lm''
and has a corresponding prediction function, e.g.\ ``predict''.

\begin{Schunk}
\begin{Sinput}
> me <- ModelEvaluation()
> rModel <- Regression(lm, predict)
\end{Sinput}
\end{Schunk}

Note, that we could also use the provided FastRegression 
object, which is a fast implementation of a ridge regression.

\begin{Schunk}
\begin{Sinput}
> rModel <- FastRegression()
\end{Sinput}
\end{Schunk}

We then create a feature selection object, namely a TwinScan
object, which aims to minimizes the average
error of predictions made with a feature set.

\begin{Schunk}
\begin{Sinput}
> fs <- TwinScan(averageError, min)
\end{Sinput}
\end{Schunk}

Now, we select a feature set from the training data using the
above objects.

\begin{Schunk}
\begin{Sinput}
> featureList <- selectFeatures(fs, trainingData, me, randomize = TRUE, 
+     rModel)
> featureList
\end{Sinput}
\begin{Soutput}
[1] 1 2 3 4 5
\end{Soutput}
\end{Schunk}

Lets create a model using the selected feature set.

\begin{Schunk}
\begin{Sinput}
> featureSet(trainingData) <- featureList
> rModel <- learnFromDataset(rModel, trainingData)
\end{Sinput}
\end{Schunk}

\section{Model Testing}

Before we come to the actual confidence estimation, lets test
our model on the test data. We assign the feature set
of the training data to the test data and predict the
responses of the test data.

\begin{Schunk}
\begin{Sinput}
> featureSet(testData) <- featureList
> predictions <- predictDataset(rModel, testData)
> predictions
\end{Sinput}
\begin{Soutput}
             [,1]
 [1,] -0.12633007
 [2,]  0.05029106
 [3,]  0.96222107
 [4,]  0.88265502
 [5,] -0.01338626
 [6,] -0.16452070
 [7,] -0.31711196
 [8,]  0.64604206
 [9,]  1.30421336
[10,] -0.41184359
[11,]  1.74850039
[12,] -0.06082061
[13,] -1.51800586
[14,]  0.14934645
[15,] -0.08763298
[16,] -0.94604161
[17,] -0.04629694
[18,] -0.99593038
[19,]  1.65450202
[20,]  1.28570963
[21,]  0.97761312
[22,] -1.16113762
[23,] -0.75595752
[24,] -0.88318289
[25,] -1.97354000
[26,]  0.48561524
[27,]  0.57489831
[28,] -0.34784174
[29,]  0.16369287
[30,]  1.71971332
[31,]  0.09097077
[32,] -0.60592559
[33,]  0.07625206
[34,] -0.31804901
[35,]  0.48852976
[36,]  0.33610229
[37,]  1.96057032
[38,]  0.61331678
[39,] -0.83325330
[40,]  0.93586990
\end{Soutput}
\end{Schunk}

In contrast, the real response values are:

\begin{Schunk}
\begin{Sinput}
> getResponses(testData)
\end{Sinput}
\begin{Soutput}
 [1] -0.095274343 -0.348808298  1.422912533  1.337572529  0.498226183
 [6]  0.802668849 -0.245927507  0.729828014  1.082307465  0.188889423
[11]  1.613101889  0.297076505 -1.451797119 -0.386881926 -0.033540275
[16] -1.146083146  0.160198548 -1.396756382  2.423148935  1.477409733
[21]  1.260573023 -0.661840301 -0.556130160 -0.934713379 -2.174491607
[26]  0.693936671  0.415853753  0.275802280 -0.091780822  2.343349362
[31] -0.301392953 -1.833809161  0.009856587 -0.619404877  0.163229116
[36]  0.521050458  2.704518761  0.623759866 -0.403960291  1.129643713
\end{Soutput}
\end{Schunk}

To evaluate the model, lets calculate the average squared error
of the predicted responses.

\begin{Schunk}
\begin{Sinput}
> averageError(getResponses(testData), predictions)
\end{Sinput}
\begin{Soutput}
[1] 0.1876864
\end{Soutput}
\end{Schunk}

Since we normalized the responses the average squared error,
the real average squared error is calculated as follows.

\begin{Schunk}
\begin{Sinput}
> realResponses <- unscaleVector(testData, getResponses(testData))
> realPredictions <- unscaleVector(testData, predictions)
> averageError(realResponses, realPredictions)
\end{Sinput}
\begin{Soutput}
[1] 4.168221
\end{Soutput}
\end{Schunk}

To get a more precise view on the performance of the model,
we can also start a five-fold nested cross-validation.

\begin{Schunk}
\begin{Sinput}
> evaluateInCV(me, data, rModel, 5, random = FALSE, fs)
\end{Sinput}
\begin{Soutput}
[1] 5.415507
\end{Soutput}
\end{Schunk}

The returned value is the average result of the quality
function, assigned at the moment of creating the ModelEvaluation
object, the average squared error in our case.

\section{Confidence Estimation}

To predict confidence scores and confidence intervals in
addition to our plain predicted responses, we first
create ConfidenceEstimator object. In our case, we use
ConfidenceEstimatorNNErrors, which estimates the
confidence of each prediction by calculating the
average sqaured error the the $m$ nearest neighbors.

\begin{Schunk}
\begin{Sinput}
> ce <- ConfidenceEstimatorNNErrors(rModel, trainingData)
\end{Sinput}
\end{Schunk}

To train our confidence estimator, we have to provide
predictions on the training data. In addition, we have
to decide whether we want to optimize our estimator
on the training dataset or want to use a kernel
estimate instead. We choose to optimize the estimator
since it often leads to better results

\begin{Schunk}
\begin{Sinput}
> predictionsTrainingData <- predictDataset(rModel, trainingData)
> ce <- create(ce, optimize = TRUE, predictionsTrainingData)
\end{Sinput}
\end{Schunk}

Now, the estimator is trained and can be used to predict
the confidences of our predictions.

\begin{Schunk}
\begin{Sinput}
> confidences <- estimate(ce, testData, predictions)
> confidences
\end{Sinput}
\begin{Soutput}
         [,1]        [,2]        [,3]
 [1,] 0.92500 -0.47355372  0.14878158
 [2,] 0.27500 -0.67788918  0.73424720
 [3,] 0.98750  0.61499743  1.15931931
 [4,] 0.40000  0.32563771  1.49431389
 [5,] 0.16250 -0.86610124  0.70254853
 [6,] 0.16250 -1.01723569  0.55141409
 [7,] 0.88125 -0.66832359 -0.04020124
 [8,] 0.94375  0.29296567  0.92553921
 [9,] 0.03125  0.36932665  2.09110391
[10,] 0.27500 -1.14002383  0.27211255
[11,] 0.05625  0.83663238  2.53106517
[12,] 0.08125 -1.00721667  0.71190908
[13,] 0.98750 -1.86522950 -1.32090763
[14,] 0.93750 -0.20186517  0.42625718
[15,] 0.71250 -0.47325338  0.32123428
[16,] 0.91875 -1.29127126 -0.67182949
[17,] 0.21875 -0.82846430  0.67445787
[18,] 0.55000 -1.39969845 -0.53717820
[19,] 0.44375  1.19061621  2.21388330
[20,] 0.37500  0.72647815  1.94739927
[21,] 0.33125  0.37817166  1.64786663
[22,] 0.82500 -1.53172946 -0.87905405
[23,] 0.93125 -1.10517515 -0.47994633
[24,] 0.30000 -1.58518819 -0.20705683
[25,] 0.94375 -2.32661639 -1.69404285
[26,] 0.63750  0.06788423  0.92733649
[27,] 0.05000 -0.34272437  1.35854453
[28,] 0.25000 -1.09565318  0.34198696
[29,] 0.71250 -0.22192754  0.57256013
[30,] 0.05000  0.80209064  2.50335954
[31,] 0.49375 -0.34108898  0.63123631
[32,] 0.08125 -1.55232165  0.16680411
[33,] 0.29375 -0.63229697  0.75433565
[34,] 0.37500 -0.87728049  0.34364063
[35,] 0.30000 -0.21347554  1.16465583
[36,] 0.92500 -0.01112135  0.61121395
[37,] 0.05000  1.04294764  2.74421654
[38,] 0.48125  0.18125703  1.15358232
[39,] 0.98750 -1.18047694 -0.63615506
[40,] 0.75625  0.55932333  1.22446781
\end{Soutput}
\end{Schunk}

For each instance in the test set, the estimator returned
a score and a lower and an upper bound of an $80\%$
confidence interval.

To analyze the quality of our confidence estimation on the
training data, we calculate the correlation of the prediction
errors with the width of the corresponding confidence intervals

\begin{Schunk}
\begin{Sinput}
> evaluateConfidenceEstimates(getResponses(testData), predictions, 
+     confidences)
\end{Sinput}
\begin{Soutput}
[1] 0.4269676
\end{Soutput}
\end{Schunk}

Depending on the random artificial dataset, we see different
results.  On average the correlation is positive.
If we would repeat everything with
a dataset of say $1000$ instances, the correlation would very likely
increase since confidence estimation is usally much more stable on
large datasets.

An alternative way of looking at the enrichment of predictions with a
low error is by calculating the confidence associated prediction improvement.
The prediction improvement shows by what percentage the average
squared error is reduced if we consider only the $20\%$ predictions
with the smalles confidence interval.
Therefore, we calculate the average squared error of the top $20\%$
predictions and normalize it by the error on all predictions.

\begin{Schunk}
\begin{Sinput}
> interval_widths <- confidences[, 3] - confidences[, 2]
> top20 <- which(interval_widths <= sort(interval_widths)[length(interval_widths) * 
+     0.2])
> top20Error <- averageError(getResponses(testData)[top20], predictions[top20])
> top20Error
\end{Sinput}
\begin{Soutput}
[1] 0.08984874
\end{Soutput}
\begin{Sinput}
> overallError <- averageError(getResponses(testData), predictions)
> overallError
\end{Sinput}
\begin{Soutput}
[1] 0.1876864
\end{Soutput}
\begin{Sinput}
> (1 - top20Error/overallError) * 100
\end{Sinput}
\begin{Soutput}
[1] 52.12826
\end{Soutput}
\end{Schunk}


Finally, we can evaluate the estimated confidence intervals 
by plotting the intervals and the predicted instances~\ref{fig:fig1}.


\begin{Schunk}
\begin{Sinput}
> plotConfidenceIntervals(ce, getResponses(testData), predictions, 
+     confidences[, 1])
\end{Sinput}
\end{Schunk}
\begin{figure}
\begin{center}
\includegraphics{confReg-fig1}
\end{center}
\caption{Predicted score-based confidence interval and predicted instances.}
\label{fig:fig1}
\end{figure}

We expect the confidence interval to be smaller for large
confidence scores. We also expect about $80\%$ of the
instances within the interval. However, especially for
small datasets, often less than the expected number of
instances is within the interval borders.


\end{document}


